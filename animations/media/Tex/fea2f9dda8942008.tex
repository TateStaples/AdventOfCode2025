\documentclass[preview]{standalone}
\usepackage[english]{babel}
\usepackage{amsmath}
\usepackage{amssymb}
\begin{document}
\begin{align*}
\text{Left: crossings} = \lfloor \text{mag} / 100 \rfloor \text{ (if mag > pos)}
\end{align*}
\end{document}